\documentclass[]{deedy-resume-openfont}
 
\begin{document}
     
%%%%%%%%%%%%%%%%%%%%%%%%%%%%%%%%%%%%%%
%
%     Profile
%
%%%%%%%%%%%%%%%%%%%%%%%%%%%%%%%%%%%%%%
  
\namesection {Anindya} {Kanti Mitra}\infoline{\href{mailto:anindya15d12m@gmail.com}{anindya15d12m@gmail.com} | +919903068420}\\\vspace{4pt}
\linksline{\descript{\href{http://github.com/anindyamitra15}{\color{blue} GitHub }}{\descript{ | }}\descript{\href{http://www.linkedin.com/in/anindyamitra15}{\color{blue} Linkedin }}}\underlineheader{}
%%%%%%%%%%%%%%%%%%%%%%%%%%%%%%%%%%%%%%
%
%     Education
%
%%%%%%%%%%%%%%%%%%%%%%%%%%%%%%%%%%%%%%
\section{Education}
\raggedright

\runsubsection{Kalyani Government Engineering College}\hspace*{\fill}  \location{August 2019 - August 2023}\\
\descript{Bachelor of Technology Electronics and Communication Engineering}\hspace*{\fill}\location{Kalyani, Nadia, West Bengal}\\
GPA: 9.16\\
\sectionsep
%%%%%%%%%%%%%%%%%%%%%%%%%%%%%%%%%%%%%%
%
%     Experience
%
%%%%%%%%%%%%%%%%%%%%%%%%%%%%%%%%%%%%%%
\section{Experience}
\runsubsection{Mind Webs Ventures}\descript{| IoT and EC Engineer}\hfill \location{Work from home, West Bengal | July 2020 – Present}
  \begin{tightemize}
  \item IoT and EC Engineer, also does Backend (Node-Express-MongoDB) and Frontend (Angular) development in non-production projects
\end{tightemize}
\sectionsep
\runsubsection{Adben Industries Pvt. Ltd.}\descript{| PCB Designer}\hfill \location{Work from home, West Bengal | Jun 2021 – Present}
  \begin{tightemize}
  \item PCB Designing for Embedded Systems. Built few multi-layer PCBs on KiCAD under non-disclosure.
\end{tightemize}
\sectionsep
\runsubsection{KGEC Robotics Society}\descript{| Board of director}\hfill \location{Kalyani, Nadia, West Bengal | April 2020 – Present}
  \begin{tightemize}
  \item I have been a member since my first year of study in KGEC. I have participated in Flipkart Grid 3.0 under the team name \textquotesingle{}Monokrome\textquotesingle{}, under the club. I am currently managing all the activities under KGEC Robotics Society.
\end{tightemize}
\sectionsep
\runsubsection{DSC KGEC}\descript{| Technical Lead}\hfill \location{Kalyani, Nadia, West Bengal | December, 2019 – Present}
  \begin{tightemize}
  \item I worked in the domain of IoT and Electronics. I was the maintainer of the project Cleanurge under KSOC. I have built projects on Socket IO, ESP8266, SIM800L GPRS module, HTTP and AJAX on embedded systems.
\end{tightemize}
\sectionsep

%%%%%%%%%%%%%%%%%%%%%%%%%%%%%%%%%%%%%%
%
%     Skills
%
%%%%%%%%%%%%%%%%%%%%%%%%%%%%%%%%%%%%%%
\section{Skills}
\raggedright
\begin{tabular}{p{5cm}p{13.5cm}}
\descript{Programming Languages} & {\location{Embedded C/C++ \textbar{} JS/TS \textbar{} Python}} \\
\descript{Libraries/Frameworks} & {\location{Firebase, Node-Express, Angular, Arduino}} \\
\descript{Tools / Platforms} & {\location{VS Code, Platform IO, STM32IDE, Arduino IDE, KiCAD, Postman}} \\
\descript{Databases} & {\location{Mongo Database}} \\
\end{tabular}
\sectionsep
%%%%%%%%%%%%%%%%%%%%%%%%%%%%%%%%%%%%%%
%
%     Projects
%
%%%%%%%%%%%%%%%%%%%%%%%%%%%%%%%%%%%%%%
\section{Projects / Open-Source}
\raggedright
  

\runsubsection{\large{Flipkart Grid 3.0}}
\descript{| \href{http://github.com/anindyamitra15/monokrome-grid}{\color{purple}Link}}\hfill \location{MQTT, Python, Embedded C++, TinkerCAD, Arduino Framework, Platform IO}\\
We built an centralized Open CV based maze-solver to control multiple bots with unique ArUco identifiers. The IoT bots, connected to Wi-Fi, had to unload a package at a specific site as per the Maze predicted by a central camera whose real-time footage was fed to an open-cv program based on python. MQTT was used for Realtime communication with the robots and the Python application in the same wifi network.\\
\sectionsep


\runsubsection{\large{Angular Pump}}
% \descript{| \href{}{Link}}\\
Angular based Progressive Web App which connects with Firebase Realtime Database to control a pump using a 30A capable DIY IoT switch based on ESP8266.\\
\sectionsep


\runsubsection{\large{Envion}}
% \descript{| \href{}{Link}}\\
A project under Mind Webs. It is an RFID based access control using IoT for use in institutions or rooms in offices.
I wrote the entire firmware for the ESP32 RFID Scanner module. I wrote most of the frontend code on Angular to show the data in tabular dashboard. I have also worked on the Backend a little bit.\\
\sectionsep

\ 
\end{document}